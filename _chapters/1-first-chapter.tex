\chapter{This is the first chapter}\label{chap:first}

\inlineminitoc

\noindent In this chapter I'll show some of the features of the \texttt{uninathesis} documentclass.

\section{Typography \& math}
Some text here. \(\varphi,\phi,\psi\vDash M U k\)
Here \texttt{true type}. 
\textsc{here Small Caps}. 
\textsf{here sans serif}. 
\emph{here italics}.
\textbf{\emph{here bold italics}}.
\textbf{\textsf{bold sans}}.
\textsf{\emph{Italic sans}}.
Here we cite \citeauthor{dijkstra1972humbleprogrammer} and \citeauthor{lamport1982proving}
who wrote \cite{dijkstra1972humbleprogrammer,lamport1982proving} respectively.
\blindmathpaper

\section{Lists}
In this section you can see how lists look like.
\subsection{Itemize}
\blinditemize
\subsection{Enumerate}
\blindenumerate
\subsection{Description}
\blinddescription
\subsection{Custom enumerate}
\begin{enumerate}[label=(\roman*)]
    \item foo;
    \item bar;
    \item foobar.
\end{enumerate}
\subsection{Inline enumerate}
You can also write inline enumerates as follows:
\begin{enumerate*}[label=(\roman*)]
    \item first item;
    \item second item;
    \item third and last item.
\end{enumerate*}

\section{Tables}
Classic booktabs tables as in Table \ref{tab:table}. \blindtext
\begin{table}
    \begin{center}
      \caption{Table using booktabs.}
      \label{tab:table}
      \begin{tabular}{llr}
        \toprule % <-- Toprule here
        \textbf{Value 1} & \textbf{Value 2} & \textbf{Value 3}\\
        $\alpha$ & $\beta$ & $\gamma$ \\
        \midrule % <-- Midrule here
        1 & 1110.1 & a\\
        2 & 10.1 & b\\
        3 & 23.113231 & c\\
        \bottomrule % <-- Bottomrule here
      \end{tabular}
    \end{center}
\end{table}

\section{Algorithms}
Algorithm environment is styled to be consistent with booktabs (same heading and bottomline). \blindtext[2]
\begin{algorithm}
    \caption{Box alignment procedure}\label{alg:padding}
    \begin{algorithmic}[1]
        \Statex \textbf{signature} $\textsc{BoxAlign}$ $CSA\times CSA \to CSA\times CSA$
        \Statex \textbf{ensure} The returned CSA are box-compatible
        \Function{$\textsc{BoxAlign}$}{$\mathscr{M},\mathscr{F}$}
            % I cut a whole part of the algorithm; It doesn't make much sense now!
            \State $(\mathscr{M}^\prime,\mathscr{F}^\prime)\gets(\mathscr{M},\mathscr{F})$
            \ForAll{$(b_m,b_f)\in B_{\mathscr{M}^\prime}\times B_{\mathscr{F}^\prime}$}
                \State $\left(\beta_{\mathscr{M}^\prime},\beta_{\mathscr{F}^\prime}\right)\gets(\varepsilon,\varepsilon)$
                \For{$0\le i < |\beta_{\mathscr{M}^\prime}(b_m)|$} 
                    \State $(\mathscr{A}_\mathscr{M},\mathscr{A}_\mathscr{F})\gets\textsc{BoxAlign}(\beta_{\mathscr{M}^\prime}(b_m)_i,\beta_{\mathscr{F}^\prime}(b_f)_i)$
                    \State $\beta_{\mathscr{M}^\prime}(b_m)\gets\beta_{\mathscr{M}^\prime}\cdot\mathscr{A}_\mathscr{M}$
                    \State $\beta_{\mathscr{F}^\prime}(b_f)\gets\beta_{\mathscr{F}^\prime}\cdot\mathscr{A}_\mathscr{F}$
                \EndFor
            \EndFor
            \State \textbf{return} $(\mathscr{M}^\prime,\mathscr{F}^\prime)$
        \EndFunction
    \end{algorithmic}
\end{algorithm}


\section{Listings}
Listings provided by the lstlistings package. Example shown in Listing \ref{lst:code}. \blindtext[2]
\begin{lstlisting}[language=Python,float,caption=Python example,label={lst:code},basicstyle=\ttfamily,frame=b,framextopmargin=.2ex]
    import numpy as np
     
    def incmatrix(genl1,genl2):
        m = len(genl1)
        n = len(genl2)
        M = None #to become the incidence matrix
        VT = np.zeros((n*m,1), int)  #dummy variable
     
        #compute the bitwise xor matrix
        M1 = bitxormatrix(genl1)
        M2 = np.triu(bitxormatrix(genl2),1) 
\end{lstlisting}


\section{Listings, Algorithm and Table: consistent styling}
The section title and Figure \ref{fig:figure} are pretty self-explanatory.\Blindtext
\begin{figure}\centering
\begin{minipage}[t]{.3\textwidth}
\begin{algorithm}[H]
    \caption{Test}
    \begin{algorithmic}[1]
        \State $z\gets 1+1$
        \State $z\gets 1+2$
    \end{algorithmic}
\end{algorithm}
\end{minipage}\hfill
\begin{minipage}[t]{.3\textwidth}%
    \vspace{.7em}
    \begin{lstlisting}[language=Python,caption=ex]
    int foo;
    foo=1;
    \end{lstlisting}
\end{minipage}\hfill%
\begin{minipage}[t]{.3\textwidth}
\begin{table}[H]
    \begin{tabular}{ll}
        \toprule
        foo & bar \\
        \midrule 
        1 & 2 \\
        \bottomrule
    \end{tabular}
\end{table}
\end{minipage}
\caption{Algorithm, code and table side by side}\label{fig:figure}
\end{figure}